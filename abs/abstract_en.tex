\begin{abstract}%

Under the wave of artificial intelligence, there are more and more automatic warehouses and stores.
For example, Amazon Go, an automatic store in Seattle, is known for its convenience: Customers can pick products and leave directly without checking out with store clerk. In Amazon Go, picked products will be automatically detected and paid by credit card. It saves lots of time for standing in line.
The other example is Online shopping. Company receives customer's Online orders, collects products in warehouse, and the shipping company delivers the goods to the customers. Many companies have started to develop automation of warehouse to improve the efficiency for collecting products. In these scenarios, robots need to depend on the semantic label on products to execute picking and placing tasks.
Picking tasks have had important progress in Amazon Picking Challenge, but placing tasks reamin challenging. For example, in an automatic store, products need to be placed neatly on the shelf, and customers can find products easily. More, when classifying products in warehouse, barcode on the products needs to face barcode scanner so it can be classified, shipped on the conveyor, and packed to specific box.
There are some challenges for pose-aware placing tasks: 1)  Semantic label and geometry on the products need to be considered jointly. 2) The occlusion in cluttered environment makes detection and manipulation hard. More, there are no unified methods to evaluate the performance for robotic manipulation. To solve these problems, I improve ~\cite{peterthesis}, and evaluate the system. The paper contributions are 1): Offer open source dataset which includes image, labels for object, barcode, and brandname. 2) Develop cooperative dual-arm active manipulation system and prove its capibility for solving the problems. Finally, I analyze the failure case for the future improvement.

\end{abstract}
