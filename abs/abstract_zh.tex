\begin{abstractzh}
在人工智慧的浪潮下,有愈來愈多的倉庫與商店走向無人化,如:位於西雅圖的 AmazonGo 無人商店,主打顧客挑選商品完就可以帶著商品離開,不須交由
店員結帳付款,而是自動辨識顧客挑選 的商品進行信用卡扣款,省去了等待排隊的時間;而網路購物也是相當便利,由廠商針對顧客網路訂單從倉儲進行集貨並由
貨運公司進行運輸到客戶手中,其中許多廠商為了提高集貨的效率已經著手於倉儲的自動化。在這些情境中,機器人需要根據商品上的語意標示進行自動夾取與放置任務。
機器人夾取任務在亞馬遜機器人競賽已有重大的突破,但放置的議題卻鮮有人研究。例如無人商店需要商品整齊排列於貨架上,讓顧客能清楚的找尋商品。此外,在倉庫中集貨時,也需要將商品上的條碼面對條碼機,系統才能分類商品並藉由輸送帶傳輸商品到指定箱子打包。
特定姿態放置任務,主要有幾項困難:1)物品上的語意標示與物品幾何需同時考慮 2)在雜亂的環境中物品互相遮蔽,導致難以辨識與操作。此外針對機器人操作的相關議題,目前並無一個統一的方法去評估機器人操作系統的性能,為了解決這些問題,我改良~\cite{peterthesis},並進行驗證。論文主要貢獻為1)提供開源的商品語意資料集,其中包含影像及物體、商標、條碼標註 2)開發主動式操作雙手臂協作系統。
藉由商品語意資料集證明此系統能有效解決上述提到之問題,並提出與分析失敗案例作為未來改善的依據。

\end{abstractzh}
