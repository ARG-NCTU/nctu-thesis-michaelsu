\chapter{結論與未來研究方向}
\label{chapter:conclusions}

\section{結論}
本研究利用語意標注(品牌文字)做為線索,對物體進行基於品牌文字之夾取可行性預測,設計商品語意資料庫作為訓練與測試,並開發主動式操作系統達到特定姿態夾取放置任務。這份成果將可被應用於無人商店與無人倉庫:商品上架任務、條碼朝上自動分類結帳等等,節省倉庫與商店勞力成本。

\subsection{商品語意資料集}
本研究針對本研究所選取的20類商品蒐集資料,建置了首個以物品上語意(品牌文字、條型碼)定義物品姿態的商品語意資料集。此資料集分為訓練集以及測試集。可用來訓練具旋轉差異性之品牌文字語意切割模型,搭配本研究所開發的夾取可行性預測系統,可達到基於品牌文字預測物品姿態、以及夾取可行性預測之功能。此外測試集有效分類場景複雜性,並可有效評估系統在各場景操作方式。

\subsection{物品夾取可行性預測}
以具旋轉差異性之品牌文字還有物品作為線索,並考量兩組夾爪的物理特性,預測2指夾爪與吸盤夾爪夾取物品可行性。此系統使用兩個語意切割模型(品牌文字語意切割模型、商品語意切割模型)作為輸入,並結合點雲分析,找出適合夾取點與並進行優先排序。根據實驗二,可證明在此系統最後預測出之夾取點,對於吸盤夾爪以及2指夾爪都有很高的成功率。

\subsection{主動式操作系統}
主動式操作系統,打破傳統單手臂的作法,參考物品夾取可行性預測系統,並決定最後行動的策略。此系統以兩階段夾取方法,解決原本baseline方法只能在品牌文字可視以及無雜亂環境下才能執行的問題,在真實世界測試集中,有不錯的成效。但若要用於實際的上架系統,未來得考慮把目標物品種類增加至數千種商品。此外,系統也需要升級為移動式操作機器人,才能將應用範圍覆蓋至整間商店。

\section{未來方向}

\subsection{提升規模至上千種物品}
一間超商動輒有上千種商品,因此若要實務應用,勢必得提升規模。在整個系統中最費時間的一環為蒐集資料集,且以目前的系統,每有新的商品便必須蒐集一份新的資料,更不論商品每季都會更新包裝。因此未來將朝向不用現實世界資料集,以虛擬世界資料集訓練模型,好處是虛擬資料集可在短時間內快速蒐集,且無須人力。但目前的虛擬資料集在一定程度上訓練出來的模型仍無法直接應用於現實世界,因此這將是未來改善的目標。此外現今使用的全卷積網路也有類別上的限制,此網路較適用於50類以下的語意切割,若類別提升為上千樣,效率將十分差勁,且可能難以收斂。因此也需要進行更換為如SSD ~\cite{liu2016ssd}、YOLO ~\cite{redmon2016you}網路模型等。

\subsection{移動式雙手臂協作機器人}
系統目前為定點的自動上架,雖有不錯的成效,但考量到現實環境的商店中無法只靠雙手臂便涵蓋到所有架子操作空間,勢必得賦予機器人場域移動能力。因此主動式操作系統除了考慮手臂之間運動空間外,也將考慮機器人可於空間內移動之範圍。此外,目前系統放置一個物品約需40秒,速度是無法符合應用需求的,因此將來希望藉由平行運算、減少手臂移動路徑去減少執行時間,目標達到平均5秒可完成放置物品。
