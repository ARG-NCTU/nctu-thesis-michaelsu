\chapter{文獻回顧}
\label{chapter:relate-work}

\section{亞馬遜拾取/機器人挑戰(Amazon Picking/Robotics Challanges 2015-2017)}
亞馬遜拾取挑戰 (Amazon Picking Challenge)、亞馬遜機器人挑戰 (Amazon
Robotics Challenge) 是由亞馬遜公司所創辦的大型競賽,主要目的為在
大型倉儲中商品可能高達上千種,層層排列的大型貨架,如以人力來依照訂單
集貨,不僅需要查看商品存放的位置還必須爬上貨梯進行搬運,倘若使用機器
人自動化地拾取商品,能夠大量縮短 集貨的時間。由於真實的倉儲的環境使得
機器人能活動的空間有限,且必須使機器人能夠自動辨認大量的商品種類,抓
取並放置正確的位置,對於倉儲自動化有相當大的推進。在上架任務中,物體
姿態估計是十分重要的一環,以六個自由度之物體姿態主要利用預先建制的三
維 CAD 模型,與深度攝影機所獲得的點雲(Point Cloud)進行點雲匹配如 ICP
(Iterative Closest Point) ~\cite{pomerleau2013comparing},
需以點雲資訊匹配三維模型,將實際物體的點雲從
深度攝影機所得到的原始點雲加以分離,才能獲得較好的效果,利用物體偵測
方法為將二維候選框資訊投影至三維得到物體的三維點雲,有相當大的可能引
入背景點雲進而影響 ICP 匹配結果,而採用物體語意分割(Semantic
Segmentation)的方法能夠將物體輪廓較精確的標記,將二維像素分類結果投影
至三維後能得到較單純的物體點雲,Amazon Picking Challenge 2016 競賽中
MIT-Princeton 團隊於提出的視覺系統~\cite{zeng2016multi}採用了物體語意分割與 ICP 點雲匹配的
方法。

\section{主動式視覺與主動式操作}
~\cite{sunderhauf2018limits}提及機器人視覺與電腦視覺領域的主要差異為:機器人視覺最後產出的為行動
(action)而非電腦視覺的資訊(information),電腦視覺或機器學習常以既有資
料集進行訓練,並著重於資料集的準確度提升,但機器人視覺則與環境互動,特
別關注:主動式機器人視覺(active vision)、主動式操作(active manipulation)。主
動式機器人視覺可透過機器人去控制移動攝影機,改變其位置與觀察視角,藉此去得到更多的環境資訊去改善優化機器人對環境的感知能力,改善系統的感
知信心指數、解決感知上的歧義、最小化被遮蔽的、反射等問題。主動式操作則是進一步利用會去改變場景去幫助感知的效果,例如:機器手臂可以去移動
被遮擋的物品去得到被隱藏的資訊。
這樣的概念被應用於這些研究中: \cite{atanasov2014nonmyopic} \cite{doumanoglou2016recovering} \cite{malmir2017deep}採取主動式機器人視覺的概念,採用``next-best viewpoint''的策略去改善感知的信心指數,並應用於物件偵測上。
 \cite{zeng2018robotic} 則偏向更高階的主動式操作,採取``grasp-first-then-recognize''策略去改善雜亂環境中糟糕的感知效果。
本研究就是藉由主動式操作,對於商標文字被遮蔽的物件進行操作,進而找出商標物件並正確進行上架。



\section{夾取成功率預測}

\section{雙機械手臂協作}

\section{機器人操作領域之基準訓練測試資料集}
