\usepackage{color}
\usepackage{latexsym}
\usepackage{amsmath,amssymb}
\usepackage{amsthm}
\usepackage{multirow}
\usepackage{bm}
\usepackage{epstopdf}
\usepackage{verbatim}

\usepackage{calc}
\usepackage{longtable}

\usepackage{graphicx}
\newtheorem{definition}{Definition}
\newtheorem{proposition}{Proposition}
\newtheorem{lemma}{Lemma}
\newtheorem{theorem}{Theorem}
\newtheorem{corollary}{Corollary}
\newtheorem{construction}{Construction}
\newtheorem{goal}{Goal}
\newtheorem{claim}{Claim}
\newtheorem{fact}{Fact}
\newtheorem{conjecture}{Conjecture}

%%%%%%%%%%%%%%%%%%%% Graphics %%%%%%%%%%%%%%%%%%%%
\graphicspath{{figures/}}

\newcommand{\onefiguren}[3]{
  \begin{figure}[tbp] 
  %\begin{minipage}{0.5
  \begin{center}
  \centering
  \includegraphics[width=\textwidth, keepaspectratio=true]{#1}
  \vspace{-4mm} %-3
  \caption{{\footnotesize #2}}
  \label{#3}
  %\vspace{-4mm} %-3
  \end{center}
  %\end{minipage}
  \end{figure}
  \vspace{7mm}
}
\newcommand{\onefigure}[5]{
  \begin{figure}[tbp] 
  %\begin{center}
	\centering  
  \includegraphics[height=#2,width=#3,keepaspectratio=true]{#1}
  \vspace{-4mm} %-3
  \caption{{\footnotesize #4}}
  \label{#5}
  \vspace{-4mm} %-3
  %\end{center}
  \end{figure}
}

\newcommand{\onefigurecap}[6]{ %used for long-caption figures.
  \begin{figure}[tbp] 
  %\begin{center}
	\centering  
  \includegraphics[height=#2,width=#3,keepaspectratio=true]{#1}
  \vspace{-4mm} %-3
  \caption[#4]{{\footnotesize #5}} %#4: caption title, #5: full caption
  \label{#6}
  \vspace{-4mm} %-3
  %\end{center}
  \end{figure}
}

\newcommand{\twofigures}[6]{
\begin{figure*}[t!]
\begin{center}
\begin{tabular}{c c}
\hspace{-5mm} \includegraphics[height=#3,width=#4,keepaspectratio=true]{#1} 
  & \includegraphics[height=#3,width=#4,keepaspectratio=true]{#2} \\
\end{tabular}
\vspace{-4mm} %-4
\caption{{\footnotesize #5}}
\label{#6}
\vspace{-4mm} %-2
\end{center}
\end{figure*}
}

\newcommand{\twofigurescap}[7]{ %used for long-caption figures.
  \begin{figure*}[t!]
\begin{center}
\begin{tabular}{c c}
\hspace{-5mm} \includegraphics[height=#3,width=#4,keepaspectratio=true]{#1} 
  & \includegraphics[height=#3,width=#4,keepaspectratio=true]{#2} \\
\end{tabular}
\vspace{-4mm} %-4
\caption[#5]{{\footnotesize #6}}
\label{#7}
\vspace{-4mm} %-2
\end{center}
\end{figure*}
}

% single column
\newcommand{\twofigurestight}[6]{
\begin{figure}[t!]
\begin{center}
\begin{tabular}{c c}
\hspace{-5mm} \includegraphics[height=#3,width=#4,keepaspectratio=true]{#1} 
  & \includegraphics[height=#3,width=#4,keepaspectratio=true]{#2} \\
\end{tabular}
\vspace{-4mm}
\caption{{\footnotesize #5}}
\label{#6}
\vspace{-2mm}
\end{center}
\end{figure}
}

\newcommand{\threefigures}[7]{
\begin{figure*}[t!]
\begin{center}
\begin{tabular}{c c c}
\hspace{-5mm} \includegraphics[height=#4,width=#5,keepaspectratio=true]{#1} \hspace{-3mm}
  & \hspace{-1mm} \includegraphics[height=#4,width=#5,keepaspectratio=true]{#2} \hspace{-3mm}
  & \hspace{-1mm} \includegraphics[height=#4,width=#5,keepaspectratio=true]{#3} \\
\end{tabular}
%\vspace{-4mm}
\caption{{\footnotesize #6}}
\label{#7}
%\vspace{-2mm}
\end{center}
\end{figure*}
}

\newcommand{\threefigurescap}[8]{ %for long caption
\begin{figure*}[t!]
\begin{center}
\begin{tabular}{c c c}
\hspace{-5mm} \includegraphics[height=#4,width=#5,keepaspectratio=true]{#1} \hspace{-3mm}
  & \hspace{-1mm} \includegraphics[height=#4,width=#5,keepaspectratio=true]{#2} \hspace{-3mm}
  & \hspace{-1mm} \includegraphics[height=#4,width=#5,keepaspectratio=true]{#3} \\
\end{tabular}
%\vspace{-4mm}
\caption[#6]{{\footnotesize #7}}
\label{#8}
%\vspace{-2mm}
\end{center}
\end{figure*}
}

\newcommand{\threefigurestight}[7]{
\begin{figure*}[t!]
\begin{center}
\begin{tabular}{c c c}
\hspace{-10mm} \includegraphics[height=#4,width=#5,keepaspectratio=true]{#1} \hspace{-5mm}
  & \hspace{-5mm} \includegraphics[height=#4,width=#5,keepaspectratio=true]{#2} \hspace{-5mm}
  & \hspace{-5mm} \includegraphics[height=#4,width=#5,keepaspectratio=true]{#3} \\
\end{tabular}
%\vspace{-4mm}
\caption{{\footnotesize #6}}
\label{#7}
%\vspace{-2mm}
\end{center}
\end{figure*}
}


%%%%%%%%%%%%%%%%%%%% Math %%%%%%%%%%%%%%%%%%%%
\def\C{{\mathbb C}}  % complex
\def\Q{{\mathbb Q}}  % rationals
\def\Z{{\mathbb Z}}  % integers
\def\N{{\mathbb N}}  % integers
\def\R{{\mathbb R}}  % reals
\def\P{{\mathbb P}}  % probability
\def\1{{\mathbb I}}  % indicator
\def\Ee{{\mathbb E}} % expectation
\def\Bc{{\mathcal B}}
\def\Cc{{\mathcal C}}
\def\Dc{{\mathcal D}}
\def\Ec{{\mathcal E}}
\def\Ic{{\mathcal I}}
\def\Kc{{\mathcal K}}
\def\Lc{{\mathcal L}}
\def\Nc{{\mathcal N}}
\def\Pc{{\mathcal P}}
\def\Rc{{\mathcal R}}
\def\Sc{{\mathcal S}}
\def\Tc{{\mathcal T}}
\def\Uc{{\mathcal U}}
\def\Vc{{\mathcal V}}
\def\Xc{{\mathcal X}}
\def\Var{{\textrm{\bf Var}}}
\def\diag{{\textrm{diag}}}
\def\Unif{{\textrm{Unif}}}
\newcommand{\me}{\mathrm{e}}  % math e
\newcommand{\mi}{\mathrm{i}}  % math i
%% upright \pi not easy since not provided in most math fonts
\newcommand{\dif}{\mathrm{d}}  % differential operator
\newcommand{\vect}[1]{\bm{#1}} % bold for vectors
\newcommand{\rv}[1]{\bm{#1}}   % bold for random variables
%\newcommand{\mat}[1]{\mathsf{#1}} % san serif for matrix
\newcommand{\tmat}[2][rrrrrrrrrr]{\left[  
\begin{array}{#1} 
  #2 \\ 
\end{array} 
\right]^{\scriptstyle {tr}}}

\newcommand{\mat}[2][rrrrrrrrrr]{\left[ 
\begin{array}{#1} 
  #2 \\ 
\end{array}
\right]\}

%%%%%%%%%%%%%%%%%%%% Source Code Environments %%%%%%%%%%%%%%%%%%%%
\newenvironment{mycode}
{\begin{list}{}{\setlength{\leftmargin}{1em}}\item\scriptsize\bfseries}
{\end{list}}

\newenvironment{mybigcode}
{\begin{list}{}{\setlength{\leftmargin}{1em}}\item\small\sffamily}
{\end{list}}

\newenvironment{mytinycode}
{\begin{list}{}{\setlength{\leftmargin}{1em}}\item\tiny\bfseries}
{\end{list}}
